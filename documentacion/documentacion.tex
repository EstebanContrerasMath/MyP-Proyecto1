\documentclass{article}

%Opciones de idioma
\usepackage[spanish]{babel}

%Hiperv�nculos
\usepackage{hyperref}
\hypersetup{
	colorlinks=true,
	linkcolor=blue,
	urlcolor=cyan,
}

\title{Modelado y Programaci\'on \\ Proyecto 1}
\author{C\'esar Hern\'andez Cruz}
\date{}

\begin{document}

\maketitle

\begin{abstract}
El Proyecto 1 de la clase Modelado y Programaci\'on consiste
en un chat, implementado como una aplicaci\'on de escritorio.   Este
chat consta de dos partes, un servidor, y un cliente, que se
comunican mediante un protocolo de red determinado en el curso.

El objetivo de este documento es describir mi implementaci\'on de
dicha aplicaci\'on.
\end{abstract}

\section{El proyecto}
La aplicaci\'on a desarrollarse consta de dos partes, un servidor y un 
cliente.   El servidor recibe conexiones de red de de diversos clientes y
atiende las solicitude que estos realizan, e.g., identificarse, enviar mensajes
a una sala de chat general, crear salas de chat, invitar a otros usuarios a
dichas salas, enviar mensajes directos a otros usuarios, cambiar de estado,
solicitar la lista de usuarios conectados, etc.   Existe un protocolo previamente
determinado para que el cliente pueda comunicarse con el servidor, y a
su vez, pueda procesar los mensajes que recibe, ya sea de otros usuarios,
o en respuesta a sus solicitudes.

El servidor est\'a implementado como una aplicaci\'on que se ejecuta en la
terminal, mientras que el cliente es una aplicaci\'on de escritorio con una
interfaz gr\'afica.


\section{Lenguaje de implementaci\'on}

El lenguaje que eleg\'i para implementar mi proyecto es
\href{https://golang.org/}{Go}.   Con la restricci\'on de tiempo para
aprender un nuevo lenguaje de programaci\'on, al mismo tiempo
desarrollar por primera vez un proyecto desde cero y hacer por
primera vez una interfaz gr\'afica, me interesaba utilizar un lenguaje
que fuera nuevo para mi, y que cumpiera con algunas restricciones:
que tuviera manejo autom\'atico de memoria, que tuviera una sintaxis
sencilla, pero que fuera robusto para llevar a cabo un proyecto
mediano.   Descart\'e Python porque, aunque cumple con las dos
primeras propiedades, en el pasado he trabajado con \'el y hab\'ia
notado que al no ser compilado, hay errores que son muy dif\'iciles
de detectar (en mi limitada experiencia), lo que lo hace no cumplir
con mi tercer requerimiento.

Me encontr\'e con que Go cuenta con una comunidad de usuarios
donde es f\'acil encontrar apoyo para aquellos que empezamos con
el lenguaje.   Adem\'as, tiene una sintaxis amigable, y tiene algunas
estructuras interesante, como los
\href{https://tour.golang.org/concurrency/2}{canales}, dise\~nados
para facilitar la concurrencia.   Tambi\'en, sus herramientas para
compilar, correr pruebas y ejecutar los programas, me parecieron
atractivas para facilitar dichas tareas y poder concentrarme en la
implementaci\'on.   Tras realizar el
\href{https://tour.golang.org/welcome/1}{tour de go}, qued\'e
convencido de utilizar este lenguaje.

En retrospectiva, creo que lo que menos me convence del lenguaje
son las bibliotecas para las interfaces gr\'aficas.  Eleg\'i gotk, un
``binding'' para gtk, cuya documentaci\'on (y ejemplos) deja mucho
que desear.   Eleg\'i esta biblioteca pues es ampliamente utilizada,
y es f\'acil encontrar ejemplos, que aunque est\'an en otros lenguajes
(principalmente C, Python y Ruby), es relativamente f\'acil traducirlos
a Go.

Utilic\'e dos versiones de Go:
\begin{itemize}
	\item go version go1.10.3 darwin/amd64 (MacOS)
	
	\item go version go1.10.1 linux/amd64 (Ubuntu)
\end{itemize}


\section{C\'omo obtener y ejecutar el programa}

El c\'odigo de mi proyecto se encuentra en mi
\href{https://github.com/Japodrilo/MyP-Proyecto1}{repositorio de github},
mismo que puede ser clonado con el comando
\begin{center}
\texttt{\$ git clone https://github.com/Japodrilo/MyP-Proyecto1.git}
\end{center}
El lenguaje Go tiene algunas restricciones respecto a la organizaci\'on
de las carpetas, y aunque no todas son est\'andar, la ruta principal
para todos los proyectos con Go debe de ser la misma para que las
herramientas de compilaci\'on y ejecuci\'on funcionen correctamente,
esta ruta es conocida como
\href{https://golang.org/doc/code.html}{\texttt{GOPATH}}
y est\'a asociada a una variable que 	especifica el lugar de nuestro
``workspace''.

En particular, la ruta para este proyecto debe de ser
\begin{center}
\texttt{GOPATH/src/github.com/Japodrilo/MyP-Proyecto1/}
\end{center}

La \'unica dependencia para este proyecto es la biblioteca para
interfaces gr\'aficas \href{https://github.com/gotk3}{gotk3}, que
puede instalarse en Ubuntu con el comando
\begin{center}
\texttt{\$ go get github.com/gotk3/gotk3/gtk}
\end{center}
y que a su vez tiene las siguientes dependencias
\begin{itemize}
	\item GTK 3.6-3.16
	
	\item GLib 2.36-2.40
	
	\item Cairo 1.10 o 1.12
\end{itemize}
que pueden instalarse en Ubuntu con el comando
\begin{center}
\texttt{\$ sudo apt-get install libgtk-3-dev libcairo2-dev libglib2.0-dev}
\end{center}
En caso de requerir instalarlo en otro sistema operativo, la
documentaci\'on oficial puede consultarse
\href{https://github.com/gotk3/gotk3/wiki#installation}{aqu\'i}.
La ruta de gotk3 debe de ser
\begin{center}
\texttt{GOPATH/src/github.com/gotk3/gotk3}
\end{center}
y la variable \texttt{GOBIN} debe de estar definida para que los
archivos binarios se guarden autom\'aticamente en dicha ruta.

Una vez con ambas variables definidas, los archivos binarios
pueden generarse con el comando
\begin{center}
\texttt{GOPATH/src/github.com/Japodrilo/MyP-Proyecto1/go install ./...}
\end{center}
mismos que se guardar�n en la ruta dada por \texttt{GOBIN}.
Tambi�n, el cliente puede correrse con el comando
\begin{center}
\texttt{GOPATH/src/github.com/Japodrilo/MyP-Proyecto1/cmd/cliente/go run cliente.go}
\end{center}
y el servidor con el comando
\begin{center}
\texttt{GOPATH/src/github.com/Japodrilo/MyP-Proyecto1/cmd/servidor/go run servidor.go puerto}
\end{center}
Las clases en el modelo contienen pruebas unitarias que corren con el comando
\begin{center}
\texttt{GOPATH/src/github.com/Japodrilo/MyP-Proyecto1/go test ./...}
\end{center} sin embargo, todas las pruebas fallan por omisi\'on.

Cabe se\~nalar que hay problemas conocidos con gotk3, y en particular,
hubo una computadora con Ubuntu en la que no pude reproducir la instalaci\'on,
y la interfaz gr\'afica no funcion\'o como se esperaba.   Algunos problemas
comunes se discuten \href{https://github.com/gotk3/gotk3/issues}{aqu\'i}.

\section{Interfaz gr\'afica}

La interfaz gr\'afica es muy sencilla.   Cuenta con un men\'u principal con
cuatro submen\'us.  El primero cuenta con las opciones para conectarse
y desconectarse.   Todas las opciones est\'an apagadas cuando el cliente
no est\'a conectado al servidor, excepto ``Conectarse'' y ``Cerrar pesta\~na
actual''.   Hay un submen\'u llamado ``Estado'', en el que las opciones
simplemente cambian el estado en el que el usuario se encuentra.   El
submen\'u pesta\~nas s\'olo tiene la opci\'on de cerrar la pesta\~na actual.
Finalmente, el men\'u ``Sala'' tiene las opciones autodescriptivas ``Crear''
e ``Invitar'', adem\'as de la opci\'on ``Mis salas'', donde se pueden abrir
pesta\~nas para las conversaciones de las salas a las que el usuario
pertenece.

\section{Problemas conocidos}

Tuve que implementar la actualizaci\'on de la lista de usuarios con ``polling'',
lo que (creo) provoca un problema al desconectar el cliente y volverlo a
conectar en la misma ventana.   A veces, el cliente indica que el nombre de
usuario que se quiere utilizar est\'a ocupado, aunque no sea as\'i.

Es posible cerrar el di\'alogo con el que el usuario se identifica con el servidor.
No es recomendable hacerlo, pero en caso de hacerlo, puede volverse a
abrir al volver a introducir los datos de conexi\'on.

\end{document}